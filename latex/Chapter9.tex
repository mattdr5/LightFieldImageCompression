\chapter{Conclusione e Sviluppi futuri}
La conclusione dello studio evidenzia l'importanza cruciale della scelta del dataset e del codec nella compressione dei light field images, con rilevanti impatti sulle prestazioni complessive. L'analisi ha mostrato variazioni significative nell'efficacia della compressione in base al dataset utilizzato, con l'ordine dei frame che può influenzare la qualità e le performance, infine, la decisione tra codec lossy e lossless è fondamentale.
\\
\\
Il codec Prores ha dimostrato di avere un interessante equilibrio tra il rapporto di compressione, la perdita di informazioni e la qualità visiva. Nonostante il suo rapporto di compressione potrebbe non essere ottimo rispetto ad alcuni codec altamente efficienti in termini di riduzione delle dimensioni, il ProRes compensa questo aspetto con eccellenti indici SSIM e PSNR, nonostante la sua natura lossy.
\\
D'altra parte, codec come MPEG4 e FLV1 dimostrano un equilibrio ben calibrato in tutte le fasi, presentando ottimi rapporti di compressione insieme a qualità visiva e fedeltà strutturale notevoli. Questi codec si distinguono per essere efficienti in termini di spazio di archiviazione senza compromettere eccessivamente la qualità dell'immagine, nostante la loro natura lossy.
\\
\\
Questi esempi evidenziano la complessità delle scelte nel campo della compressione video, poiché diversi codec possono eccellere in modi differenti.
\\
\\
Per lo sviluppo futuro, potrebbe essere interessante esplorare ottimizzazioni specifiche per determinati dataset e valutare l'evoluzione di nuovi approcci o codec che possano offrire prestazioni superiori. Un'altra direzione potenziale potrebbe essere quella di esaminare nuove metodologie nell'ambito dei Light Field, poiché l'adattamento continuo delle tecniche di compressione alle immagini light field rimane un ambito di ricerca promettente.
