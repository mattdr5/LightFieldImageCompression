\chapter{Lavori correlati}

Per la compressione dei ligth field ci siamo rifatti allo studio di un progetto passato svolto da nostri colleghi nell'università di Salerno\cite{LigthFieldCompression}.
\\
\\
I loro risulati si basano sulla compressione dei light field (che sono rappresentati tramite una serie immagini) in file di tipo video, così da sfruttare vari codec che si usano per comprimere le immagini in video.
\\
Precisamente i nostri colleghi hanno utilizzato i seguenti codec:
\begin{itemize}
    \item HVEC (lossy e lossless)
    \item VP9 (lossy e lossless)
    \item AV1 (lossy e lossless)
    \item HuffYUV (lossless)
    \item UT Video (lossless)
    \item FFV1 (lossless)
\end{itemize}
La nostra idea è quella di proseguire la strategia che i nostri colleghi hanno proposto andando ad espandere il loro studio già presente.
